\documentclass[a4paper,14pt,russian]{article}

\usepackage{hyperref} % ссылки
\usepackage[warn]{mathtext} % русские буквы в формулах
\usepackage{extsizes}
\usepackage{cmap} % для кодировки шрифтов в pdf
\usepackage[T2A]{fontenc}
\usepackage[utf8]{inputenc}
\usepackage[russian,english]{babel}

\DeclareSymbolFont{T2Aletters}{T2A}{cmr}{m}{it} % курсивные русские буквы в формулах

\usepackage{graphicx} % для вставки картинок
\usepackage{amssymb,amsfonts,amsmath,amsthm} % математические дополнения от АМС
\usepackage{indentfirst} % отделять первую строку раздела абзацным отступом тоже
\usepackage[usenames,dvipsnames]{color} % названия цветов
\usepackage[most]{tcolorbox} % ещё одна библиотека для работы с цветом
\usepackage{makecell}
\usepackage{multirow} % улучшенное форматирование таблиц
\usepackage{ulem} % подчеркивания
\usepackage[top=0.8in, bottom=0.8in, left=0.625in, right=0.625in]{geometry} % Нормальные поля
%\usepackage{fancyhdr} % Контитулы

% Цвета для гиперссылок
\definecolor{linkcolor}{HTML}{799B03} % цвет ссылок
\definecolor{urlcolor}{HTML}{799B03} % цвет гиперссылок
\hypersetup{pdfstartview=FitH,  linkcolor=linkcolor,urlcolor=blue, colorlinks=true}

\usepackage{listings} % для вставки кода
\usepackage{caption} % рамка вокруг кода
\DeclareCaptionFont{white}{\color{white}} %% это сделает текст заголовка белым
%% код ниже нарисует серую рамочку вокруг заголовка кода.
\DeclareCaptionFormat{listing}{\colorbox{gray}{\parbox{\textwidth}{#1#2#3}}}
\captionsetup[lstlisting]{format=listing,labelfont=white,textfont=white}

\lstset{ %
	language=C,                 % выбор языка для подсветки (здесь это С)
	basicstyle=\small\sffamily, % размер и начертание шрифта для подсветки кода
	numbers=left,               % где поставить нумерацию строк (слева\справа)
	numberstyle=\tiny,           % размер шрифта для номеров строк
	stepnumber=1,                   % размер шага между двумя номерами строк
	numbersep=5pt,                % как далеко отстоят номера строк от подсвечиваемого кода
	backgroundcolor=\color{white}, % цвет фона подсветки - используем \usepackage{color}
	showspaces=false,            % показывать или нет пробелы специальными отступами
	showstringspaces=false,      % показывать или нет пробелы в строках
	showtabs=false,             % показывать или нет табуляцию в строках
	frame=single,              % рисовать рамку вокруг кода
	tabsize=2,                 % размер табуляции по умолчанию равен 2 пробелам
	captionpos=t,              % позиция заголовка вверху [t] или внизу [b] 
	breaklines=true,           % автоматически переносить строки (да\нет)
	breakatwhitespace=false, % переносить строки только если есть пробел
	escapeinside={\%*}{*)},   % если нужно добавить комментарии в коде
	keywords={true, false, catch, function, return, null, if, in, while, do, else, then, break, for, from, import, auto, int, long, bool, const, void, double},
	keywordstyle=\color{blue}\bfseries,
	keywords=[2]{}
	keywordstyle=[2]\color{green}\bfseries,
	identifierstyle=\color{black},
	sensitive=false,
	comment=[l]{//},
	morecomment=[s]{/*}{*/},
	commentstyle=\color{purple}\ttfamily,
	stringstyle=\color{red}\ttfamily,
	morestring=[b]',
	morestring=[b]"
}

%\pagestyle{fancy}
\author{Сапожников Денис}

\newcommand{\divisible}{\mathop{\raisebox{-2pt}{\vdots}}} %делится
\newcommand{\Mod}[1]{\ (\mathrm{mod}\ #1)} %нормальный mod

%% Теоремы
\theoremstyle{plain}
\newtheorem{theorem}{Теорема}
\newtheorem{lemma}[theorem]{Лемма} 
\newtheorem{proposition}[theorem]{Предложение}
\newtheorem{corollary}[theorem]{Следствие}
\newtheorem{claim}[theorem]{Утверждение}

\newtheorem*{claim*}{Утверждение}
\newtheorem*{theorem*}{Теорема}
\newtheorem*{lemma*}{Лемма}
\newtheorem*{corollary*}{Следствие}

%% Определения
\theoremstyle{definition}
\newtheorem{definition}[theorem]{Определение}
\newtheorem{problem}[theorem]{Задача}
\newtheorem{problems}[theorem]{Задачи}

\newtheorem*{definition*}{Определение}
\newtheorem*{problem*}{Задача}
\newtheorem*{problems*}{Задачи}
\newtheorem*{fact*}{Факт}

%% Замечания и примеры
\theoremstyle{remark}
\newtheorem{example}[theorem]{Пример}
\newtheorem{examples}[theorem]{Примеры}
\newtheorem{remark}[theorem]{Замечание}

\newtheorem*{example*}{Пример}
\newtheorem*{remark*}{Замечание}

\newenvironment{enumerate*}{\begin{enumerate}\setlength{\leftskip}{-1em}}{\end{enumerate}}
\newenvironment{itemize*}{\begin{itemize}\setlength{\leftskip}{-1em}}{\end{itemize}}

\newcommand{\ma}{\displaystyle}

\begin{document}
	\title{Отчёт по задаче Рюкзак}
	\author{Сапожников Денис}
	\date{}
	\maketitle
	Я написал удобный класс Knapsack со множество методов:
	\begin{itemize}
		\item solve\_recover\_dp$(W)$ --- решает рюкзак методом ДП и восстанавливает ответ техникой divide-and-conqueror. $O(nW)$ времени, $O(n)$ памяти.
		\item solve\_recover\_meet\_in\_the\_middle$(W)$ --- решает рюкзак методом meet-in-the-middle. $O(n2^{\frac{n}{2}})$ времени и памяти.
		\item solve\_recover\_branch\_bound$(W)$ --- решает рюкзак методом Branch\&Bound. $O(2^n)$ времени в худшем случае. Верхняя граница считается $\frac{1}{2}$-аппроксимацией. Порядок предметов -- по стоимости. Порядок ветвей -- если взять предмет + $\frac{1}{2}$-аппроксимация больше, чем не взять + аппроксимация, то сначала идём в ветвь, где берём предмет, иначе -- сначала в ветвь, где не берём.
		\item solve\_recover\_approximation\_half$(W)$ --- половинная аппроксимация с лекции.
		\item solve\_recover\_approximation\_slow$(W, \varepsilon)$ --- первый медленный алгоритм $\varepsilon$-аппроксимации, работает за $O(n \log n + \frac{n^2}{\varepsilon})$, взят с лекции.
		\item solve\_recover\_approximation\_ibarra$(W, \varepsilon)$ --- алгоритм Ибарры $\varepsilon$-аппроксимации, за $O(n \log n + \frac{n}{\varepsilon^2})$, взят с лекции.
		\item solve\_recover\_on\_center\_optimization$(W, kL, kR, middle\_solver, *args)$. 
		
		Предположительно, оптимальный ответ устроен очень просто, если отсортировать все предметы по удельной стоимости $\frac{c}{w}$: $[\underbrace{1, 1, \ldots, 1}_{левая\ часть}, \underbrace{1, 0, 0, 1, \ldots, 1}_{центральная\ часть}, \underbrace{0, 0, \ldots, 0}_{правая\ часть}]$. Пусть $m_{half}$ --- граница $\frac{1}{2}$-аппроксимации, тогда выберем границы центральной части как $[m_{half} - kL; m_{half} + kR]$ и отправим этот набор предметов в \\
		$middle\_solver(W - w_{left}, *args)$. 
	\end{itemize}
	
	\textbf{Результаты экспериментов:}
	
	Потестовые результаты экспериментов доступны в приложенном csv-файле. 
	\begin{itemize}
		\item Техника Branch\&Bound работала очень долго даже при $n=30$, поэтому этот эксперимент вышел неудачным. Возможно, стоило хорошо аппроксимировать начальное приближение, чтобы отсечь много веток сразу, но я не успел попробовать.
		
		\item Как оказалось, медленный алгоритм аппроксимации давал результат лучше, чем алгоритм Ибарры при одном и том же $\varepsilon$ (и даже при больших $\varepsilon$). 
		
		\item Алгоритм с оптимизацией блока по центру вышел лучше всех, и стал выдавать хорошие результаты даже при meet-in-the-middle с параметрами $kL=15, kR=30$. 
		
		Техника Branch\&Bound работала долго и в центральной оптимизации. 
		
		Но вот медленная аппроксимация выдала очень хорошие результаты, и итоговое решение выделяет блок по центру с параметрами $kL=100, kR=500$ и аппроксимирует его с точностью $0.05\%$. Кроме того, было замечено, что при меньших границах иногда результат бывает лучше, поэтому запускается ещё и центральная оптимизация с параметрами $kL=50, kR=250$ и точностью $0.05\%$. Ну и для уверенности в себе ещё запускается алгоритм Ибарры с точностью $5\%$ (меньшие значения работали так же по приближению, но дольше, поэтому я решил выдать больше времени на центральную оптимизацию) и центральная оптимизация с meet-in-the-middle$(kL=15, kR=30)$. Среди всех подходов берётся лучший.
	\end{itemize}

	\textbf{id посылки} --- 49864540.
\end{document}
