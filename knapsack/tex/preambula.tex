\documentclass[a4paper,14pt,russian]{article}

\usepackage{hyperref} % ссылки
\usepackage[warn]{mathtext} % русские буквы в формулах
\usepackage{extsizes}
\usepackage{cmap} % для кодировки шрифтов в pdf
\usepackage[T2A]{fontenc}
\usepackage[utf8]{inputenc}
\usepackage[russian,english]{babel}

\DeclareSymbolFont{T2Aletters}{T2A}{cmr}{m}{it} % курсивные русские буквы в формулах

\usepackage{graphicx} % для вставки картинок
\usepackage{amssymb,amsfonts,amsmath,amsthm} % математические дополнения от АМС
\usepackage{indentfirst} % отделять первую строку раздела абзацным отступом тоже
\usepackage[usenames,dvipsnames]{color} % названия цветов
\usepackage[most]{tcolorbox} % ещё одна библиотека для работы с цветом
\usepackage{makecell}
\usepackage{multirow} % улучшенное форматирование таблиц
\usepackage{ulem} % подчеркивания
\usepackage[top=0.8in, bottom=0.8in, left=0.625in, right=0.625in]{geometry} % Нормальные поля
%\usepackage{fancyhdr} % Контитулы

% Цвета для гиперссылок
\definecolor{linkcolor}{HTML}{799B03} % цвет ссылок
\definecolor{urlcolor}{HTML}{799B03} % цвет гиперссылок
\hypersetup{pdfstartview=FitH,  linkcolor=linkcolor,urlcolor=blue, colorlinks=true}

\usepackage{listings} % для вставки кода
\usepackage{caption} % рамка вокруг кода
\DeclareCaptionFont{white}{\color{white}} %% это сделает текст заголовка белым
%% код ниже нарисует серую рамочку вокруг заголовка кода.
\DeclareCaptionFormat{listing}{\colorbox{gray}{\parbox{\textwidth}{#1#2#3}}}
\captionsetup[lstlisting]{format=listing,labelfont=white,textfont=white}

\lstset{ %
	language=C,                 % выбор языка для подсветки (здесь это С)
	basicstyle=\small\sffamily, % размер и начертание шрифта для подсветки кода
	numbers=left,               % где поставить нумерацию строк (слева\справа)
	numberstyle=\tiny,           % размер шрифта для номеров строк
	stepnumber=1,                   % размер шага между двумя номерами строк
	numbersep=5pt,                % как далеко отстоят номера строк от подсвечиваемого кода
	backgroundcolor=\color{white}, % цвет фона подсветки - используем \usepackage{color}
	showspaces=false,            % показывать или нет пробелы специальными отступами
	showstringspaces=false,      % показывать или нет пробелы в строках
	showtabs=false,             % показывать или нет табуляцию в строках
	frame=single,              % рисовать рамку вокруг кода
	tabsize=2,                 % размер табуляции по умолчанию равен 2 пробелам
	captionpos=t,              % позиция заголовка вверху [t] или внизу [b] 
	breaklines=true,           % автоматически переносить строки (да\нет)
	breakatwhitespace=false, % переносить строки только если есть пробел
	escapeinside={\%*}{*)},   % если нужно добавить комментарии в коде
	keywords={true, false, catch, function, return, null, if, in, while, do, else, then, break, for, from, import, auto, int, long, bool, const, void, double},
	keywordstyle=\color{blue}\bfseries,
	keywords=[2]{}
	keywordstyle=[2]\color{green}\bfseries,
	identifierstyle=\color{black},
	sensitive=false,
	comment=[l]{//},
	morecomment=[s]{/*}{*/},
	commentstyle=\color{purple}\ttfamily,
	stringstyle=\color{red}\ttfamily,
	morestring=[b]',
	morestring=[b]"
}

%\pagestyle{fancy}
\author{Сапожников Денис}

\newcommand{\divisible}{\mathop{\raisebox{-2pt}{\vdots}}} %делится
\newcommand{\Mod}[1]{\ (\mathrm{mod}\ #1)} %нормальный mod

%% Теоремы
\theoremstyle{plain}
\newtheorem{theorem}{Теорема}
\newtheorem{lemma}[theorem]{Лемма} 
\newtheorem{proposition}[theorem]{Предложение}
\newtheorem{corollary}[theorem]{Следствие}
\newtheorem{claim}[theorem]{Утверждение}

\newtheorem*{claim*}{Утверждение}
\newtheorem*{theorem*}{Теорема}
\newtheorem*{lemma*}{Лемма}
\newtheorem*{corollary*}{Следствие}

%% Определения
\theoremstyle{definition}
\newtheorem{definition}[theorem]{Определение}
\newtheorem{problem}[theorem]{Задача}
\newtheorem{problems}[theorem]{Задачи}

\newtheorem*{definition*}{Определение}
\newtheorem*{problem*}{Задача}
\newtheorem*{problems*}{Задачи}
\newtheorem*{fact*}{Факт}

%% Замечания и примеры
\theoremstyle{remark}
\newtheorem{example}[theorem]{Пример}
\newtheorem{examples}[theorem]{Примеры}
\newtheorem{remark}[theorem]{Замечание}

\newtheorem*{example*}{Пример}
\newtheorem*{remark*}{Замечание}

\newenvironment{enumerate*}{\begin{enumerate}\setlength{\leftskip}{-1em}}{\end{enumerate}}
\newenvironment{itemize*}{\begin{itemize}\setlength{\leftskip}{-1em}}{\end{itemize}}